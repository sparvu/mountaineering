
\documentclass[xcolor=dvipsnames,aspectratio=1610]{beamer}
\usetheme{boxes}
\usecolortheme{whale}
\definecolor{SPM}{rgb}{0.90, 0.10, 0.25}
\usecolortheme[named=SPM]{structure}
\usepackage{multicol}
\usepackage{pgfplots}
\pgfplotsset{compat=1.15}
\usepackage{mathrsfs}
\usetikzlibrary{arrows}
\hypersetup{
  colorlinks,
  allcolors=.,
  urlcolor=blue,
}

\begin{document}


% ----------------------------------------------------------------
% Equipment
% ----------------------------------------------------------------
\section{Equipment}

\begin{frame}
\begin{center}
\Huge \textbf{Equipment}
\end{center}
\end{frame}



% ----------------------------------------------------------------
% Boots
% ----------------------------------------------------------------
\begin{frame}{Hiking and Mountaineering Boots}{}
\begin{center}
\Huge \textbf{Hiking and Mountaineering}\\~\\
\Huge \textbf{Boots}
\end{center}
\end{frame}


\begin{frame}{Hiking and Mountaineering Boots}{}
\LARGE
\textbf{Boots}\\~\\
\Large
We choose our boots based on the usage and fit. What kind of hike or 
expedition we plan: long trip, heavy backpack on a snow, bad terrain in a 
mountain or just a light walk into the forest?
\end{frame}


\begin{frame}{Hiking and Mountaineering Boots}{}
\LARGE
\textbf{Hiking boots}\\~\\
\Large
Hiking boots are made for use on rocks, mud, grass, and any other 
terrain found in the mountains. There are different type 
of hiking boots, some used for rocks some for alpine, icy 
conditions.
\end{frame}

\begin{frame}{Hiking and Mountaineering Boots}{}
\LARGE
\textbf{Mountaineering boots}\\~\\
\Large
Mountaineering or alpine boots are high-top footwear that is 
primarily designed for use in mountainous terrain, such as permanent
snowfields, glaciers, and alpine rock. They are very stiff and often 
contain a solid shank underfoot to provide a rigid sole, which is 
preferred for steep snow and ice climbing.
\end{frame}

\begin{frame}{Hiking and Mountaineering Boots}{ABCD Classification}
\LARGE
\textbf{ABCD Classification}\\~\\
\Large 
In order to be able to properly measure how efficient the hiking boots are 
for different conditions, Meindl and Hanwag have introduced the ABCD 
classification. The classification indicates the stiffness of both the shaft and 
the sole.\\~\\
A is the least rigid shoe, best suited for roads and easy terrain\\
C-D representing rock and ice climbing shoes
\end{frame}


\begin{frame}{Hiking and Mountaineering Boots}{ABCD Classification}
\LARGE
\textbf{Hiking boots - category}\\~\\
\Large
\begin{itemize}
    \item \textbf{A and A/B} - for walks and easy hikes on solid paths and in low mountain ranges
    \item \textbf{B and B/C} - for demanding mountain hikes and trekking tours on trails and via ferratas
    \item \textbf{C and D} - for off-trail summit mountain tours with rock and ice contact. Crampon resistant
\end{itemize}
\end{frame}


\begin{frame}{Hiking and Mountaineering Boots}{ABCD Classification}
\includegraphics[scale=0.28]{Images/good-boots}
\end{frame}

\begin{frame}{Hiking and Mountaineering Boots}{}
\LARGE
\textbf{Example}\\~\\
\Large
\begin{itemize}
    \item \textbf{A} hikes on gravel roads or very easy trails
    \item \textbf{B/C} steep, rocky, muddy and tree covered terrain. (Repovesi, IsoJarvi, Halti)
    \item \textbf{C/D} very steep, rocky, more demanding terrain and glaciers. (Kebnekaise, Lyngen Alps, Mont Blanc)\\~\\
\end{itemize}
\alert{\textbf{NO SPORT SHOES FOR HIKING or MOUNTAINEERING}}
\end{frame}


\begin{frame}{Hiking and Mountaineering Boots}{}
\LARGE
\textbf{Finnish conditions}\\~\\
\Large 
"In Finnish conditions, B and B/C shoes are the most suitable for hiking, 
depending on the terrain, backpack and user. As a general rule, the more weight 
you put on the shoe, the stiffer the sole should be. In more challenging terrain, 
proper ankle support is also important. Lightweight, flexible shoes are best 
suited for hikers with light loads and for easier terrain."\\~\\

See more: \href{https://varuste.net/en/how-to-choose-hiking-boots}{How to choose
hiking boots}
\end{frame}


\begin{frame}{Hiking and Mountaineering Boots}{General rules}
\includegraphics[scale=0.28]{Images/all-boots}
\end{frame}


% ----------------------------------------------------------------
% Clothing
% ----------------------------------------------------------------
\begin{frame}{Clothing}{}
\begin{center}
\Huge \textbf{Clothing}\\~\\
\Huge \textbf{Layering Basics}
\end{center}
\end{frame}


\begin{frame}{Clothing. Layering Basics}{}
\LARGE
\textbf{Clothing}\\~\\
\Large
When hiking the best approach for clothing is the ancient art of layering.
We dress in layers. This simple layering system acts as an thermostat
no matter of the weather or season. It is very simple to use and it is a tried 
and true strategy for clothing in mountains, cold or hot weather conditions. 
\end{frame}



\begin{frame}{Clothing. Layering Basics}{}
\LARGE
\textbf{The functions of each layer}\\~\\
\Large
\begin{itemize}
    \item \textbf{Base layer} underwear layer, wicks sweat off your skin
    \item \textbf{Middle layer} insulating layer, retains body heat to protect you from the cold
    \item \textbf{Outer layer} shell layer, shields you from wind and rain\\~\\
\end{itemize}
\end{frame}


\begin{frame}{Clothing. Layering Basics}
\begin{center}
\huge 
\textbf {But how do we ensure our clothing equipment is lightweight and breathable?}
\end{center}
\end{frame}


\begin{frame}{Clothing. Layering Basics}
\begin{center}
\Huge \textbf{
One Single Answer\\
\alert{Merino Wool}}
\end{center}
\end{frame}


\begin{frame}{Clothing. Layering Basics}{}
\LARGE
\textbf{Merino Wool}\\~\\
\Large
 Merino wool clothes are exceptionally lightweight and breathable, making them
 the perfect selection for layering for hiking, trekking and mountaineering.
\end{frame}


\begin{frame}{Clothing. Layering Basics}{}
\LARGE
\textbf{Merino Wool}\\~\\
\Large
Merino has the remarkable ability to regulate your body temperature so that no 
matter what activity you’re doing (or how quickly you’re doing it) you can stay 
warm during cold days and cool on hot ones. It will regulate to keep you cool in 
different conditions.
\end{frame}


\begin{frame}{Clothing. Layering Basics}{Merino Wool}
\LARGE
\textbf{Merino Wool Properties}\\~\\ 
    \begin{multicols}{3}
    \begin{itemize}
        \item Comfortable
        \item Durable
        \item Lightweight
        \item Anti-Microbial
        \item Anti-odor
        \item Sweat Wicking
        \item Insulates
        \item UV protection
        \item Fire resistant
        \item Anti-static
        \item Sustainable
        \item Biodegradable
    \end{itemize}
    \end{multicols}
\end{frame}


\begin{frame}{Clothing. Layering Basics}{Merino Wool}
\LARGE
\textbf{Staying warm is key: from the inside!}\\~\\
\Large
\begin{itemize}
    \item \textbf{Base layer} Devold Breeze, Devold Expedition (winter)
    \item \textbf{Middle layer} Devold Nibba, Devold Thermo (winter)
    \item \textbf{Outer layer} Devold Nibba, Devold Thermo (winter)
\end{itemize}
In this example I used Devold and their own Merino wool products.
\end{frame}


\begin{frame}{Clothing. Layering Basics}{Merino Wool}
\includegraphics[scale=0.28]{Images/layers}
\end{frame}


% ----------------------------------------------------------------
% Appex
% ----------------------------------------------------------------
\begin{frame}{Hiking and Mountaineering}{}
\begin{center}
\Huge \textbf{Trail Levels}\\~\\
\Huge \textbf{Hiking and Mountaineering}
\end{center}
\end{frame}

\begin{frame}[plain,noframenumbering]
\makebox[\linewidth]{\includegraphics[width=\paperwidth]{Images/hiking-levels}}
\end{frame}


\end{document}

